%% Работа с русским языком
\usepackage{cmap}			 % поиск в PDF
\usepackage{mathtext} 		 % русские буквы в формулах
\usepackage[T2A]{fontenc}	 % кодировка
\usepackage[utf8]{inputenc}	 % кодировка исходного текста
\usepackage[russian]{babel}	 % локализация и переносы

%% Пакеты для работы с математикой
\usepackage{amsmath,amsfonts,amssymb,amsthm,mathtools}
\usepackage{icomma}

%% Нумерация формул (опционально)
%\mathtoolsset{showonlyrefs=true} % показывать номера только у тех формул, на которые есть \eqref{} в тексте.
%\usepackage{leqno}               % нумерация формул слева

%% Шрифты
\usepackage{euscript}	 % шрифт "Евклид"
\usepackage{mathrsfs}    % красивый мат. шрифт

%% Некоторые полезные макросы для дебага (в случае недоверия авторам шаблона)
\makeatletter
\newcommand\thefontsize{The current font size is: \f@size pt} % пример: \section{\thefontsize}
\makeatother

%% Настройка размеров шрифтов
\makeatletter
\setlength{\headheight}{28pt}
%% TODO: мне не удалось разобраться, как грамотно подбирать второе число в 
%% \@setfontsize\*, но ряд эксппериментов показывает, что "10" выравнивает текст весьма прилично :)
\renewcommand\Huge{\@setfontsize\Huge{14pt}{10}}
\renewcommand\huge{\@setfontsize\huge{14pt}{10}}
\renewcommand\Large{\@setfontsize\Large{14pt}{10}}
\renewcommand\large{\@setfontsize\large{12pt}{10}}
\makeatother

%% Поля (геометрия страницы)
\usepackage[left=3cm,right=1.5cm,top=2cm,bottom=2cm,bindingoffset=0cm]{geometry}

%% Русские списки
\usepackage{enumitem}
\makeatletter
\AddEnumerateCounter{\asbuk}{\russian@alph}{щ}
\makeatother

%% Работа с картинками
\usepackage{caption}
\captionsetup{justification=centering} % центрирование подписей к картинкам
\usepackage{graphicx}                  % вставки рисунков
\graphicspath{{images/}{images2/}}     % папки с картинками
\setlength\fboxsep{3pt}                % отступ рамки \fbox{} от рисунка
\setlength\fboxrule{1pt}               % толщина линий рамки \fbox{}
\usepackage{wrapfig}                   % обтекание рисунков и таблиц текстом

%% Работа с таблицами
\usepackage{array,tabularx,tabulary,booktabs} % дополнительная работа с таблицами
\usepackage{longtable}                        % длинные таблицы
\usepackage{multirow}                         % слияние строк в таблице

%% Красная строка
\setlength{\parindent}{2em}

%% Интервалы
\linespread{1}
\usepackage{multirow}

%% TikZ
\usepackage{tikz}
\usetikzlibrary{graphs,graphs.standard}

%% Верхний колонтитул
\usepackage{fancyhdr}
\pagestyle{fancy}

%% Перенос знаков в формулах (по Львовскому)
\newcommand*{\hm}[1]{#1\nobreak\discretionary{}{\hbox{$\mathsurround=0pt #1$}}{}}

%% Дополнительно
\usepackage{float}   % добавляет возможность работы с командой [H] которая улучшает расположение на странице
\usepackage{gensymb} % красивые градусы
\usepackage{caption} % пакет для подписей к рисункам, в частности, для работы caption*
\usepackage{listings} % пакет для листингов с кодом
\lstset{              % настройки для лисингов с кодом
basicstyle=\small\ttfamily,
columns=flexible,
breaklines=true
}

% Hyperref (для ссылок внутри  pdf)
\usepackage[unicode, pdftex]{hyperref}

% Отступ перед первым абзацем в каждом разделе
\usepackage{indentfirst}

\DeclareMathOperator*{\argmax}{arg\,max}
\DeclareMathOperator*{\argmin}{arg\,min}

\usepackage{fancyvrb}

\lstdefinestyle{mypython}{
    language=Python,
    basicstyle=\ttfamily\small,
    keywordstyle=\color{blue},
    commentstyle=\color{gray}\itshape,
    stringstyle=\color{orange!80!black},
    showstringspaces=false,
    breaklines=true,
    frame=single,
    numbers=none,
    backgroundcolor=\color{gray!5},
    upquote=true,
    inputencoding=utf8,
    literate=
    {а}{{\cyra}}1 {б}{{\cyrb}}1 {в}{{\cyrv}}1 {г}{{\cyrg}}1 {д}{{\cyrd}}1
    {е}{{\cyre}}1 {ё}{{\cyryo}}1 {ж}{{\cyrzh}}1 {з}{{\cyrz}}1 {и}{{\cyri}}1
    {й}{{\cyrishrt}}1 {к}{{\cyrk}}1 {л}{{\cyrl}}1 {м}{{\cyrm}}1 {н}{{\cyrn}}1
    {о}{{\cyro}}1 {п}{{\cyrp}}1 {р}{{\cyrr}}1 {с}{{\cyrs}}1 {т}{{\cyrt}}1
    {у}{{\cyru}}1 {ф}{{\cyrf}}1 {х}{{\cyrh}}1 {ц}{{\cyrc}}1 {ч}{{\cyrch}}1
    {ш}{{\cyrsh}}1 {щ}{{\cyrshch}}1 {ъ}{{\cyrhrdsn}}1 {ы}{{\cyrery}}1
    {ь}{{\cyrsftsn}}1 {э}{{\cyrerev}}1 {ю}{{\cyryu}}1 {я}{{\cyrya}}1
    {А}{{\CYRA}}1 {Б}{{\CYRB}}1 {В}{{\CYRV}}1 {Г}{{\CYRG}}1 {Д}{{\CYRD}}1
    {Е}{{\CYRE}}1 {Ё}{{\CYRYO}}1 {Ж}{{\CYRZH}}1 {З}{{\CYRZ}}1 {И}{{\CYRI}}1
    {Й}{{\CYRISHRT}}1 {К}{{\CYRK}}1 {Л}{{\CYRL}}1 {М}{{\CYRM}}1 {Н}{{\CYRN}}1
    {О}{{\CYRO}}1 {П}{{\CYRP}}1 {Р}{{\CYRR}}1 {С}{{\CYRS}}1 {Т}{{\CYRT}}1
    {У}{{\CYRU}}1 {Ф}{{\CYRF}}1 {Х}{{\CYRH}}1 {Ц}{{\CYRC}}1 {Ч}{{\CYRCH}}1
    {Ш}{{\CYRSH}}1 {Щ}{{\CYRSHCH}}1 {Ъ}{{\CYRHRDSN}}1 {Ы}{{\CYRERY}}1
    {Ь}{{\CYRSFTSN}}1 {Э}{{\CYREREV}}1 {Ю}{{\CYRYU}}1 {Я}{{\CYRYA}}1
}

\lstset{style=mypython}

\usepackage{minted}