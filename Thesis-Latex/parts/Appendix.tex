\section*{Приложение}
\addcontentsline{toc}{section}{Приложение}
\label{sec:Apendix} \index{Apendix}



\subsection*{Приложение A: Запрос для LLM-судьи} \label{app:judge_prompt}
\begin{minted}[fontsize=\small, frame=single]{python}
def get_judge_prompt(question, relevant_context, reference_answer, model_answer):
    messages = []

    messages.append({"role": "system", "content": ("You are a helpful assistant "
    that scores RAG answers against the ground truth. Always respond with a JSON "
    object containing a 'comment' (string), 'is_inappropriate_refusal' (0-1), "
    "'score' (number 1-5), 'is_correct' (0-1) fields.")})

    judge_prompt = f"""
    **Question**: {question}
    **Relevant context**: {relevant_context}
    **Reference Answer**: {reference_answer}
    **Model Answer**: {model_answer}

    **Evaluation Rules**:
    1. Scoring Scale:
    - 5: Good answer (factually correct, complete)
    - 4: Mostly correct with minor issues
    - 3: Partially correct but has inaccuracies
    - 2: Mostly incorrect but contains relevant elements
    - 1: Completely wrong or irrelevant

    2. Binary Correctness:
    - 1 (Correct): Answer conveys the same meaning as reference 
    (exact wording not required)
    - 0 (Incorrect): Meaning differs from reference or contains false information

    3. Binary Refuse:
    - 1 (Inappropriate refusal): The model answer is a refusal, 
    while the reference contains a full answer
    - 0 (A normal answer or an appropriate refusal)

    4. Key Principles:
    - Focus on whether the model answers the question asked, 
    not on whether it exactly matches the reference
    - There is no need to lower the score if the model's answer 
    omits some facts that are insignificant to the question
    - If an answer seems redundant in relation to the reference, but the 
    information is contained in a relevant context, then there is no need 
    to lower the score for this
    - Ignore stylistic differences if core meaning is preserved
    - Inappropriate refusal should always have score 2 out of 5!

    Provide analysis in JSON format: {{"comment": "...", 
    "is_inappropriate_refusal": 0|1, score": 1-5, "is_correct": 0|1}}"""

    messages.append({"role": "user", "content": judge_prompt})

    return messages
\end{minted}



\subsection*{Приложение B: Запрос при генерации COT разметки} \label{app:cot_prompt}
\begin{minted}[fontsize=\small, frame=single]{python}
def get_cot_prompt(context_list, question, answer):
    context = ''
    for i, c in enumerate(context_list):
        context += f'Источник [{i+1}]:'+"\n"+c+"\n\n"

    prompt = (
        f"# Контекстная информация:\n\n{context}\n\n"
        f"# Вопрос:\n\n{question}\n"
        f"# Оригинальный ответ:\n\n{answer}\n"
        "# Задача:\n\n"
        "Твоя задача состоит в том, чтобы дать более детальный ответ на" \
        "вопрос, используя исключительно информацию из контекста.\n" \
        "Для этого последовательно порассуждай про то, какую информацию" \
        "хочет узнать пользователь и как информация из " \
        "оригинального ответа с этим соотносится.\n" \
        "В своем ответе не ссылайся на оригинальный ответ, но учти, что он " \
        "абсолютно корректен и тебе нужно лишь дать расширенную его версию.\n" \
        "Указывай номер источника, про который рассуждаешь.\n" \
        "Последний абзац должен начинаться со слова ОТВЕТ:\\n и " \
        "содержать полный ответ на поставленный вопрос.\n"
        "В нем особенно важно указывать номера использованных источников." \
        "# Твой расширенный ответ:\n\n"
    )
    return prompt
\end{minted}



\subsection*{Приложение C: конфигурации обучения} \label{app:learning_config}

\begin{table}[ht]
\centering
\caption{RU конфигурация обучения}
\fontsize{12}{14}\selectfont
\renewcommand{\arraystretch}{1.2}
\begin{tabularx}{\textwidth}{
  >{\centering\arraybackslash}p{5cm} 
  >{\centering\arraybackslash}p{10cm}
}
\toprule
\textbf{Параметр} & \textbf{Значение} \\
\midrule
Learning rate & 0.00004 \\
\midrule
Device batch size & 8 \\
\midrule
Gradient accumulation & 4 \\
\midrule
Global batch size & 32 \\
\midrule
Epoch & 1 \\
\bottomrule
\end{tabularx}
\end{table}

\begin{table}[ht]
\centering
\caption{Default WebGLM конфигурация обучения}
\fontsize{12}{14}\selectfont
\renewcommand{\arraystretch}{1.2}
\begin{tabularx}{\textwidth}{
  >{\centering\arraybackslash}p{5cm} 
  >{\centering\arraybackslash}p{10cm}
}
\toprule
\textbf{Параметр} & \textbf{Значение} \\
\midrule
Learning rate & 0.00002 \\
\midrule
Device batch size & 8 \\
\midrule
Gradient accumulation & 4 \\
\midrule
Global batch size & 32 \\
\midrule
Epoch & 1 \\
\bottomrule
\end{tabularx}
\end{table}

\begin{table}[ht]
\centering
\caption{RAFT WebGLM конфигурация обучения}
\fontsize{12}{14}\selectfont
\renewcommand{\arraystretch}{1.2}
\begin{tabularx}{\textwidth}{
  >{\centering\arraybackslash}p{5cm} 
  >{\centering\arraybackslash}p{10cm}
}
\toprule
\textbf{Параметр} & \textbf{Значение} \\
\midrule
Learning rate & 0.00002 \\
\midrule
Device batch size & 4 \\
\midrule
Gradient accumulation & 8 \\
\midrule
Global batch size & 32 \\
\midrule
Epoch & 1 \\
\bottomrule
\end{tabularx}
\end{table}

\begin{table}[ht]
\centering
\caption{Synth конфигурация обучения}
\fontsize{12}{14}\selectfont
\renewcommand{\arraystretch}{1.2}
\begin{tabularx}{\textwidth}{
  >{\centering\arraybackslash}p{5cm} 
  >{\centering\arraybackslash}p{10cm}
}
\toprule
\textbf{Параметр} & \textbf{Значение} \\
\midrule
Learning rate & 0.00001 \\
\midrule
Device batch size & 2 \\
\midrule
Gradient accumulation & 12 \\
\midrule
Global batch size & 24 \\
\midrule
Epoch & 1 \\
\bottomrule
\end{tabularx}
\end{table}
